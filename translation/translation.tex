\documentclass[UTF8]{ctexart}

\usepackage{amsmath}
\usepackage{amssymb}
\usepackage[a4paper, left=1.5cm, right=1.5cm]{geometry}

\begin{document}

\begin{abstract}
我们引入了一种新的格基约简算法,其近似保证类似于 LLL 算法,而实际性能远远超过了当前的最先进水平。我们通过在递归算法结构中迭代应用精度管理技术实现了这些结果,并展示了这种方法的稳定性。我们分析了算法的渐近行为,并展示了启发式运行时间为 $O(n^\omega (C + n)^{1+\epsilon})$,其中 $n$ 为格的维度,$\omega \in (2, 3]$ 限制了规模约简、矩阵乘法和 QR 分解的成本,$C$ 限制了输入基 $B$ 的条件数的对数。这使得在常见应用中,对于精度 $p = O(\log \|B\|_{\max})$,运行时间为 $O(n^\omega (p + n)^{1+\epsilon})$。我们的算法完全实用,我们已经发布了我们的实现。我们通过实验验证了我们的启发式方法,对许多类别的密码学格进行了广泛的基准测试,并展示了我们的算法显著优于现有实现。
\end{abstract}

\section{引言}

    \subsection{技术概览}

\section{背景}

    \subsection{记号}

    \subsection{格约化算法历史}

    \subsection{格约化的基本组成}

    \subsection{启发假设}

\section{格的轮廓和它们的应用}

    \subsection{轮廓的例子}

    \subsection{格轮廓的函数}

    \subsection{轮廓的压缩和轮廓的落差}

    \subsection{格约化的条件}


\section{改进的格约化算法}

    \subsection{格的压缩}

    \subsection{子格约化}

    \subsection{局部约化好的基的格约化}

        \subsubsection{设置约化参数 $ \alpha $}

        \subsubsection{迭代轮数的控制}

        \subsubsection{设置相似参数 $\gamma$}

    \subsection{LR约化的表现分析}

        \subsubsection{potential与轮廓落差的关系}

        \subsubsection{控制运行时间}

    \subsection{一般格基的约化}

\end{document}